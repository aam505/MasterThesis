\section{Introduction}
\label{section:intro}
VR has been shown to alter people's perceptions and bring about realistic physical and behavioural reactions to situations that are virtual and fabricated. For example, virtual environments (VEs) can decrease the perception of pain \cite{muhlberger2007pain}, aid phobia treatment \cite{parsons2008affective} and activate stereotypical social responses (\cite{dotsch2008virtual}). From arachnophobia \cite{garcia2002virtual} to fear of heights \cite{meehan2002physiological}, virtual contexts have induced realistic fear-related physiological changes. In these cases, people reacted implicitly to VEs in congruence with their real-world experience. What is more, some researchers have noted that adding a physical objects greatly increased presence and realism in a virtual pit situation \cite{meehan2002physiological}. Apart from phobias, attitudes and behaviour are the most common perceptual changes that have been studied in virtual environments. Physical performance and related perceptual incongruencies have scarcely been investigated in such contexts. A notable exception here is the work of Peña and colleagues, which explores performance changes in a Wii tennis game, varying opponents' and players' appearances \cite{pena2016see}.
\\
Virtual humans are a powerful driver for people's behaviour changes in VR. Apart from reactions to fear-inducing stimuli, most research exploring the transformative powers of VR has focused on virtual bodies. In this field of study, we identify two main themes: agency and viewpoint. Agency relates to the nature of the operator behind the virtual representation --- an actual human or an algorithm. Viewpoint refers to the point of view assumed over the virtual representations. From a first-person perspective, theories of body ownership rooted in the \textit{rubber hand illusion} \cite{botvinick1998rubber} have shown that varying one's avatar determines behavioural and perceptual changes. Many experiments done by Slater \cite{slater2014transcending,slater2006virtual}, Yee and Bailenson \cite{yee2009proteus} shed light on the \textit{transcending} power of virtual reality.  From the \textit{Proteus Effect} to theories of embodiment, numerous frameworks have been put forward to explain behavioural changes that occur when adopting virtual personas. However, it often happens that we are observers of these virtual bodies as much as we embody them.  Researchers have often overlooked the third-person perspective when exploring perceptual changes in observers. An example of this is Pan and Slater's work on interactions between males and virtual females \cite{pan2007preliminary}.
\\
Considering the focus on embodiment, we aim to investigate perceptual changes occurring in users of virtual reality from a third-person perspective. To this end, we leverage the social power of virtual bodies and design a competitive VR rope-pulling game. We look at whether different levels of perceived strength would lead users to change their performance. To inform the design of the avatars, we first ran a survey to review perceived strength and intimidation in virtual humans. After establishing a baseline for strong and weak looking avatars, we ran an empirical study in which users played tug-of-war with five avatars representing various levels of perceived strength. We used a real physical rope which maintained the same resistance across all trials. Our main hypothesis was that participants would perceive the rope as having various degrees of resistance. Furthermore, we expected them to use more force for stronger-looking avatars. To validate these assumptions, we measured the challenge and perceived force of each rope-pull instance. This set up allowed us to measure users' physical and perceived performance while seamlessly varying avatar appearances.
\\
Our results show that users did indeed perceive variations in the force acting on the rope. However, we observed notable individual variations in users' ability to succumb to this illusion and sustain it throughout the game. Furthermore, the manner in which they pulled the rope was not consistent with our assumptions. Many participants reported an expectation to use more force with stronger opponents. Despite this, our quantitative results are inconclusive and suggest users were interacting with the system in unforeseen and unrealistic ways. While most VR experiences elicit realistic responses, this has not always been the case in our study.  In the following, we present various incongruencies between quantitative data and qualitative feedback and give an overview of our findings.  Our results inform the design of realism in virtual reality encounters and shed light on challenges designers might face in order to produce and sustain presence and plausibility illusions.