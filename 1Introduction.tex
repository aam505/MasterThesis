\section{Introduction}
\label{section:intro}
VR has been shown to alter people's perceptions and bring about realistic physical and behavioural reactions to situations that are virtual and fabricated. For example, virtual environments (VEs) can decrease the perception of pain \cite{muhlberger2007pain}, aid phobia treatment \cite{parsons2008affective} and activate stereotypical social responses (\cite{dotsch2008virtual}). From spiders \cite{garcia2002virtual} to fear of heights \cite{meehan2002physiological}, virtual contexts have induces significant physiological changes such as increased heart rate or skin conductance. In these cases, people reacted implicitly to VEs in congruence with their real-world experience. What is more, for\cite{meehan2002physiological}, adding a physical object to provide haptic feedback greatly increased presence in a virtual pit situation. 
\\
Social influence is also a powerful driver for peoples behaviour change. Many experiments done by Slater \cite{slater2014transcending,slater2006virtual} and Yee together with Bailenson \cite{yee2009proteus} shed light on the \textit{transcending} power of virtual reality. Dubbed as an \textit{empathy machine}\footnote{https://www.wired.com/brandlab/2015/11/is-virtual-reality-the-ultimate-empathy-machine/} by the media, people can assume various personas in virtual environments, and experience what used to be a thought exercise in a tangible way \cite{shin2018empathy}.
\\
With its roots in the \textit{rubber hand illusion} \cite{botvinick1998rubber}, many researchers have turned to body ownership to explain physical and behaviour changes caused by VR. Christou and Michael explore the effects of avatars on performance. They created a game in which users deflected incoming objects while embodying a human or, stronger, alien avatar. Users had fewer misses and males used more force in the alien condition. 
\\
 We leverage the power of social VR and design a competitive virtual reality rope-pulling game. In our experiment, we manipulate the appearance of virtual humans to make them appear stronger or weaker.  We aim to give participants the illusion that the rope is being pulled back harder by stronger opponents. Equivalently, we expect people to feel the rope being pulled less by weaker opponents.  If participants assume these expectations, we hypothesize they will also perceive pulling harder for strong opponents and less for weaker ones. Finally, we measure their actual force pull to check if their expectation has determined a physical reaction. \\
\\
Our results show most participants detected variations in pulling. However the way in which they pulled was not consistent with our hypotheses, despite qualitative feedback. -detail In what follows we give an overview of our findings, explain the limitations of our study and give advice about using physical objects in VR to generate illusions. It appears physical illusions seem feasible even in a low-fidelity environments. However, we posit that some degree of behavioural and physical realism is desired, especially to sustain these perceptual overrides. If physical illusions can occur with such a simple setup, generating complex illusions is limited only by the creativity of the designer.
\\
Virtual reality's (VR) immersive capabilities have been successfully applied in fields such as entertainment, military training and phobia management in order to elicit real-world human experiences. Bowman and McMahan examine some of these successful VR ventures and suggest their aspired fidelity to real life is what sets them apart \cite{bowman2007virtual}. In the long term, the goal of many VR researchers and enthusiasts seems to be engaging as many of the human senses as possible to replicate real-life sensory experience with high accuracy.