\section{Related Work}
In the following, we give an overview of the technical background that informed the design of our study. 

\subsection{The Transformative Power of Avatars}
The threshold model of social influence put forward by Blascovich \cite{blascovich2002theoretical} implies that digital actors, who are perceived to be human have more influence than their computer counterparts. Both perceived agency and agency have been shown to play a role in eliciting social influence, especially for tasks such as persuasion \cite{guadagno2007virtual,fox2015avatars}. 
We use the term \textit{avatar} to refer to a user’s digital and embodied representation in virtual reality and the term \textit{agent} for computer-driven digital entities. Additionally, people have been shown to respond socially to computers even when they are not necessarily embodied in avatars, known as the Computers As Social Actors paradigm \cite{nass1994computers}. 
In a virtual reprise of a well known social-psychological study - Milgram’s obedience experiment - Slater et al. observed that people had realistic responses when shocking a virtual learner \cite{slater2006virtual}.
This occurred despite participants knowing that their activities had no effect in reality.\\
Digital self-representations have far-reaching implications from altering people's behaviour in the virtual world, to shaping perceptions in reality.
\textit{The Proteus Effect}, postulated by Yee and Bailenson, refers to how users change their behaviour, conforming to the perceived behaviour of their digital representation \cite{yee2007proteus}. 
 They observed that users in more attractive avatars got closer to confederates and showed more self-disclosure than participants with less attractive avatars. Similarly, in a negotiation task about monetary splits, people with taller avatars were more confident and did more splits in their favour, while owners of shorter avatars were more likely to accept unfair deals. \\
The Proteus Effect is framed around self-perception theory \cite{bem1972self} in a context in which users are deindividuated \cite{zimbardo1969human}. In such cases, users rely more on identity cues and behave in a way which conforms with the stereotypes of their virtually displayed body. Peña and colleagues provide an alternate explanation for this phenomenon. They suggest priming as an underlying cause of the Proteus Effect and raise concerns about users role-playing in their new identities \cite{pena2009priming}.
They measure participants' awareness about the true scope of the study and use clothing as a means for priming. Avatars were shown from a third person perspective in a 3D distributed desktop virtual environment. In their first experiments, participants were given a negotiation task to solve in groups of 3 either, having all white or black robes. Results showed that participants in black presented more aggressive intent and lower group cohesion than those in white. In their second experiment, participants had either a Ku Klux Klan (KKK), doctor or transparent avatar and were tasked with writing 2 stories. Those with a KKK type avatar wrote more aggressive stories compared with the other groups.\\
To decouple the effects of priming and embodiment, Yee and Bailenson ran a study in which users were given an attractive or unattractive avatar in an immersive virtual environment (IVE) \cite{yee2009difference}. Users were looking at themselves in a mirror inside the IVe, or, alternatively, they saw a playback of someone from before. Participants interacted with a confederate of the opposite gender who was blind to the condition, and then completed a dating website task.
The authors found that virtual embodiment resulted in more behaviour change with users. In the attractive condition, people chose more attractive dates and got closer to the confederates. In the opposite condition, users were more likely to increase their height in the dating profile.\\
Extending their work on the Proteus effect, Yee, Bailenson and colleagues found that in online communities, an avatar’s appearance was a predictor of performance. 
Furthermore, changes in behaviour have been observed to last outside of IVEs. For example, participants with tall avatars had more aggressive negotiations with confederates \cite{yee2009proteus}. The authors further  investigated this effect in a conversation between 2 opposite-gendered participants in a distributed medium \cite{van2013proteus}. In this case, the findings, were inconsistent with the Proteus effect. Females who had an unattractive avatar were reported to behave more friendly, affectionate and intimate. The authors explain these findings through the behavioural compensation effect \cite{bond1972effect}.\\
The Proteus effect does not always benefit the user. In similar experiments, women having highly sexualized avatars objectified themselves more \cite{fox2013embodiment}, embodiment in black avatars increased implicit racial bias \cite{groom2009influence}. Users were also more prone to persuasion by avatars that mimicked them \cite{bailenson2006transformed} or consistently gazed towards them \cite{bailenson2005digital}. 
More recently, virtual social exclusion had similar negative effects with real-world exclusion, leading to less prosocial behaviour outside of VR \cite{kothgassner2017real}.\\
In a series of experiments, Slater et. al put forward the idea of body semantics to explain how body ownership illusions can generate behaviour and attitudes in users \cite{slater2014transcending}. Some examples are reducing implicit racial bias for owners of black avatars or affecting the  perception of object sizes in the case of child avatars. Furthermore, the effects of reduced racial bias were observed for at least a week outside VR \cite{banakou2016virtual}. 
This paradigm differs from the Proteus Effect by attributing this change of behaviour to the generation of the body ownership illusion. While people have mostly experienced positive outcomes due to the empathy-driven by embodiment, highly stereotypical contexts may have opposite outcomes.\\
Virtual reality has shown its potential as an efficient tool for framing social and psychological studies. 
However, most research has focused on behaviour and attitude changes stemming from one's avatar. Few studies examine changes induced by the appearance of another's avatar or look at performance. In a notable example, Peña, Khan and Alexopoulos vary avatar and opponent body size in a Nintendo Wii tennis exergame and explain their findings through social comparison theory and priming \cite{pena2016see}. They found that participants with obese avatars had less physical activities than those in normal avatars and showed that this effect was mediated by the appearance of their opponents. When the opponent had a more obese avatar, participants performed less. In the same experiment with women, they found that participants made the most effort when both avatars were normal and, furthermore, the appearance of obesity decreased performance \cite{pena2014increasing}. In a series of experiments for health and behaviour change in VEs, Fox and Bailenson observed that participants made more exercise when their avatar lost or gained weight according to their movements \cite{fox2009virtual}. From a first-person perspective, Christou and Michael explore the effects of avatars on performance. They created a game in which users deflected incoming objects while embodying a human or, stronger, alien avatar. Users had fewer misses and males used more force in the alien condition \cite{christou2014aliens}. We contribute to research on performance changes in IVEs, by looking at whether participants  use more force when faced with a stronger opponent than with a weaker opponent. We implement a rope-pulling game in virtual reality and allow participants to see their hands on the rope. However, we do not vary their body representation.
 
\subsection{Embodiment}
The effects of appearance have also been studied with respect to ownership illusions. Lin and Jorg investigate the influence of six different hand models on ownership \cite{lin2016need}. In two experiments, participants played a game in which they had to block white spheres and experienced a threat scenario where a knife slashed their hand in the VE. They note that all these hands generated ownership for at least some participants, with various levels of strength. However, the effect was strongest for the realistic model and weakest for a \textit{non-anthropomorphic} bloc. Despite this, the most realistic hands do not always receive the highest ratings. In their study Argelaguet et. al \cite{ argelaguet2016role} evaluate ownership and agency for hands in 3 different realism condition that offer different degrees of freedom. In their study, the most realistic hand with the most accurate tracking was rated highest in ownership. Conversely, agency was stronger for less realistic hands. They conclude that the incongruence in ratings occurred because the highly realistic hand was often mismatched with the actual hand. This was not the case for the less realistic hands providing fewer degrees of freedom.
\\
The rubber hand illusion\cite{botvinick1998rubber} (RHI) is an experiment in psychology to demonstrate the formation of body ownership through congruent visual and tactile stimulation. Usually, a rubber hand is placed where a participant's hand should be, and both hands are stroked or tapped in a synchronized manner. The illusion created by the synchronized tactile and visual feedback generates a feeling of ownership of the fake arm for the participant. This illusion has been successfully replicated in mixed \cite{ijsselsteijn2006my} and virtual reality \cite{slater2008towards},  where researchers have also used avatars to generate feelings of ownership over virtual bodies. Slater and colleagues explore in an extensive body of work the parameters and effects of this illusion \cite{slater2009inducing}. They introduce the term, \textit{sense of embodiment} to capture the broader experience of embodiment in VR, which has three distinguishable levels: body-ownership, agency and self-location \cite{kilteni2012sense}. To evaluate the occurrence of RHI, researchers usually measure the perceived location of participants' hands and use questionnaires for subjective data. When movements are synchronized and the illusion occurs, usually there is a displacement of the perceived location closer to the rubber hand called \textit{proprioceptive drift}. \cite{rohde2011rubber}. \\
In an experiment combining embodied cognition and body ownership, Bailey, Bailenson and Casasanto \cite{bailey2016does} explore \textit{space-valence associations} as a result of mirroring participants' hand movements. They conclude that multiple senses have to be engaged in order to obtain an effect. \\

\subsection{VR Illusions}
The illusions that enable the feeling of presence have been extensively studied in literature. 
\textit{Presence} has many dimensions ranging from location, simply \textit{being there}, to various aspects of computer-mediated communication such as social presence or co-presence --- \textit{being there together} \cite{zhao2003toward}.
For IVEs, Slater and colleagues propose a separation of presence in \textit{Place Illusion} for location-related presence, and \textit{Plausibility Illusion} to denote the perceptual override that occurs when users perceive what they know cannot occur. He defines \textit{sensorimotor contingencies (SCs)} as actions users perform in order to achieve perceptual clarity. Some examples are as bending to see below, or touching a virtual object and receiving haptic feedback. He contends these VR illusions occur \textit{as a function} of possible SC. The author defines plausibility illusion as``\textit{the illusion that what is apparently happening is really happening (even though you know for sure that it is not)}'' \cite{slater2009place}. He emphasizes being the target of events is an important element in producing this illusion. Moreover, motor and visual synchronicity are essential to produce place illusion. Additionally, Slater posits that a correlation between a user's sensations and events they have not caused is important to bring about plausibility illusions (Psi). In his experiments, the author notes that Psi also occurs in low physical realism conditions, such as in the reprise of the obedience experiment \cite{slater2006virtual}. To illustrate this illusion, most of the examples refer to interactions between virtual humans, such as reacting to the gaze of avatars. However, he also mentions people's realistic responses to a virtual pit, despite their knowledge there are no such objects in real life \cite{slater1995taking}. Commercial VR applications are able to reproduce this illusion with relative easy \footnote{\url{https://store.steampowered.com/app/517160/Richies_Plank_Experience/}}. When haptic feedback is added to, the response is even more realistic, with significantly increased heart rate \cite{meehan2002physiological}. 
\\
We identify three key elements in Slater's framework: synchronicity, correlation and realism. The body is, of course, the main focal point of these illusions. The author emphasises that these illusions take place despite people having knowledge of the actual environment. However, their reactions to the environment seem to be automatic. As such, some level of perceptual override takes place in order to generate these illusions. We posit, however, that these illusions can occur even in ambiguous settings, where participants do not have knowledge of their actual environment. Considering the powerful, and mostly undocumented effect of haptic feedback for VR illusions,

\\
It seems that eliciting realistic reactions in VEs is highly dependent on sustaining body ownership and meeting users' expectations. In our study, for the tug-of-war game participants were represented by an avatar showing both of their arms. Participants used VR gloves and held a real, physical rope that corresponded to the virtual rope. The opponents did not respond to participants' pull, and there was no force activating on the rope. However, in order to meet users' expectations and to maintain realism, we used a spring and an elastic band to give some resistance when participants pulled. Furthermore, we used animations, game physics and sound to give participants continuous feedback about the state of the game. Overall, we paid particular attention to timing and synchronization in the design of the game. It has been observed that congruence between visuomotor actions gives rise to agency  \cite{kilteni2012sense} and ownership is determined by synchronized haptic and/or visual feedback. We combine and leverage these synchronicities to match the expected outcome of peoples actions. Such \textit{sorimotor contingencies} \cite{slater2009place} allow us to create a context in which VR perceptual illusions could occur. We use spring resistance to give participants some motor sensations as the rope moving towards them is something they would expect to happen in a rope-pulling game. Realistic interactions that allow visual and motor synchronization seem to contribute to the formation of these illusions \cite{slater2009place}. The ambiguous source of the motor feedback and its magnitude is what participants will have to discern. While we do not vary their appearance, we acknowledge that studies exploring a one-sided view of this interaction may fall short. However, avatar appearance is outside the scope of the present work. Despite this, we hope our findings can inform future research that looks at the relationship between user experience, ownership and realism in sustaining realistic behaviour and giving rise to more complex VR illusions.
