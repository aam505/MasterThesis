\section{Related Work}

\subsection{Body Ownership}
- what it is with virtual hand illusion
- some classifications
-studies with hands
- studies with ownership and performance

\subsection{Social VR}

The threshold model of social influence put forward by Blascovich \cite{blascovich2002theoretical} implies that digital actors, who are perceived to be human (have high agency) will have more influence than their computer counterparts. Both perceived agency and agency have been shown to play a role in social influence especially for tasks such as persuasion \cite{guadagno2007virtual,fox2015avatars}. 
We use the term \textit{avatar} to refer to a user’s own digital and embodied representation in virtual reality and the term \textit{agent} for computer-driven digital entities. 

People have been shown to respond socially to computers even when they are not necessarily embodied in avatars, known as the Computers As Social Actors paradigm \cite{nass1994computers}. 
In an virtual reprise of a well known social-psychological study - Milgram’s obedience experiment - Slater et al. observed that people had realistic responses when shocking a virtual learner \cite{slater2006virtual}.
This occurred despite knowing their activities had no real effect.

Digital self representations have far-reaching implications from altering a user’s own behavior in the virtual world to shaping perceptions in reality.
\textit{The Proteus Effect}, postulated by Yee and Bailenson, refers to how users change their own behavior conforming to the perceived behavior of their digital representation \cite{yee2007proteus}. 
 They observed that users in more attractive avatars got closer to confederates and showed more self disclosure than participants with less attractive avatars. Similarly, in a negotiation task about monetary splits, people with taller avatars were more confident and made more splits in their favor, while owners of shorter avatars were more likely to accept unfair deals. 

The Proteus Effect is framed around self-perception theory \cite{bem1972self} in a context in which users are deindividuated \cite{zimbardo1969human}. In such cases users rely more on identity cues and behave in a way which conforms with the stereotypes of their virtually displayed body. 

An alternate explanation is provided by Peña and colleagues. They suggest priming as am underlying cause of the Proteus Effect and raise concerns about users role playing in their new identities \cite{pena2009priming}.
They measure participant’s awareness about the true scope of the study and use clothing as means for priming. Avatars were shown from a third person perspective in a 3D distributed desktop virtual environment. In their first experiments, participants were given a negotiation task to solve in groups of 3 either, having all white or black robes. Results showed that participants in black presented more aggressive intent and lower group cohesion than those in white. In their second experiment, participants had either a KKK, doctor or transparent avatar and were tasked with writing 2 stories. Those employing a KKK type avatar wrote more aggressive stories compared with the other groups.

To decouple the effects of priming and embodiment, Yee and Bailenson ran a study in which users were given an attractive or unattractive avatar in an immersive virtual environment (IVE) \cite{yee2009difference}. Inside the environment, they were looking at themselves in a mirror or they saw a playback of someone from before. Participants interacted with a confederate of the opposite gender who was blind to the condition, and then completed a dating website task.
They found that virtual embodiment resulted in more behavior change with users. In the attractive condition people chose more attractive dates and got closer to the confederates. In the opposite condition, users were more likely to increase their height in the dating profile.

Extending their work on the Proteus effect, Yee, Bailenson and colleagues found that in online communities an avatar’s appearance was a predictor of performance. 
Furthermore changes in behavior lasted outside of IVEs with participants given tall avatars having more aggressive negotiations with confederates \cite{yee2009proteus}. Further, they investigated this effect in a conversation between 2 opposite-gendered participants in a distributed medium \cite{van2013proteus}. The avatars the male participants saw varied from attractive to attractive or no appearance. Females saw their avatar in a congruent condition.
The findings, however, were inconsistent with the Proteus effect. Females who had an unattractive avatar were reported to behave more friendly, affectionate and intimate. The authors explain these findings through the behavioral compensation effect \cite{bond1972effect}.

However, this phenomenon does not always benefit the user. In similar experiments, women having highly sexualized avatars objectified themselves more \cite{fox2013embodiment}, embodiment in black avatars increased implicit racial bias \cite{groom2009influence}. Users were also more prone to persuasion by avatars that mimicked them \citeor{bailenson2006transformed} or consistently gazed towards them \cite{bailenson2005digital}. 
More recently, virtual social exclusion had similar negative effects with real-world exclusion leading to less prosocial behavior outside of VR \cite{kothgassner2017real}.

In a series of experiments, Slater et. al put forward the idea of body semantics to explain how body ownership illusions can generate behavior and attitudes in users \cite{slater2014transcending}. Some examples are reducing implicit racial bias for owners of black avatars or affecting perception of object sizes in the case of child avatars. Furthermore, the effects of reduced racial bias were observed for at least a week outside VR \cite{banakou2016virtual}. 
This paradigm differs from the Proteus Effect by attributing this change of behavior to the generation of the body ownership illusion. While people have mostly experienced positive outcomes due to the empathy driven by embodiment, highly stereotypical contexts may have opposite outcomes.


Virtual reality has shown its potential as an efficient tool in framing social and psychological studies. 
While some real-world experimental designs might be unethical and prove harmful for participants, virtuality allows researchers to be transparent about their goals and provide a safer environment with fewer real-world consequences.

\subsection{Presence and Realism}



The illusions that enables to the feeling of presence have been extensively studied in literature. 
\textit{Presence} has many dimensions ranging from location, simply \textot{being there}, to various aspects of computer-mediated communication such as social presence or co-presence -\textit{being there together} \cite{zhao2003toward}.
For IVEs Slater and colleagues propose a separation of presence in \textit{Place Illusion} for location-related presence, and \textit{Plausibility Illusion} to denote the perceptual override that occurs when users percieve what they know can not occur.
Virtual body ownership is vital for both these illusions to take place successfully. 

%reveal the gap
From gender, clothing, appearance traits like attractiveness and behavioral traits like gaze and proxemics, it seems users are consistently and inconsistently influenced by other entities or their own appearance [30].  These effects arise in embodied VR but can also be observed in 3D desktop environments. The underlying mechanism is still debated with explanations raging from deindividuation coupled with behavior confirmation, self-perception theory, behavior compensation and priming. Studies allowing participants to vary their appearance are few compared to controlled  conditions and the customization habits of avatars and change are not fully understood



-slater plausaibility illusion
- how arousal increases immersion
- vr with objects or lackthereof
-why make it the same room



