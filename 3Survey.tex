
\section{Survey}
To measure perceived strength in avatar designs, we ran a survey with 17 participants (8 female), aged 22-28. Participants were recruited from the university and through snowball sampling. The survey was gender matched and each participant had to rate 18 avatar thumbnails on a 5-point likert scale measuring traits perceived from their appearance, namely strength, attractiveness, intelligence and intimidation. The order of the avatars was the same for all participants, but their thumbnails were randomly selected from an initial ordered set. The survey took 10 to 15 minutes to complete. 

\subsection{Survey Questions}
Participants were told we were interested to see how people perceive traits of avatars based on their design and looks. The goal of evaluating strength and intimidation was not made completely transparent to avoid any possible bias. Initially, we gathered demographic data like gender and age. For each avatar, participants had to answer the following questions on a 5-point likert scale from 1 - Strongly disagree to 5 - Strongly agree:

This avatar looks attractive.
This avatar looks strong.
This avatar looks intelligent. 
This avatar looks intimidating.

We are not interested in measuring intelligence, however it serves the purpose of distracting participants from the specific goal of the survey. Our main interests are strength and intimidation as these variables may have the effect of changing user performance. Attractiveness is tangentially related because of the Halo Effect [9], whereby participants could be determined to assign more positive traits to the avatars. While the order of the avatars are randomized, there could still be some effect due to ordering until users adjust the framing of their ratings. A further consideration with respect to the survey responses is agency. Since social responses can vary with agency [10],  when answering the scale on intimidation, participants may depend on plausibility [11] resulting in lower intimidation responses from thumbnails of virtual humans.

\subsection{Participants}
\subsection{Results}


\subsection{Discussion}
