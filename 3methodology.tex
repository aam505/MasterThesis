
\section{Methodology}
\subsection{Research Aims}
The aim of the experiment is to explore Q1, namely whether the appearance of an avatar can affect task performance. We frame our research in the context of a competitive strength task between two users in virtual reality (Q2). A competitive set up allows participants to face their opponents directly and inspect their appearance in order to establish possible game strategies.
\subsection{Hypothesis}
\subsection{Study Design}
We investigate change of strength performance through a VR rope pulling game. Players are told they are testing a rope pulling VR game and they will be asked for pointers afterwards how to improve the game and make it more fun. 
Each participant are told they will face several different players in VR with whom they will compete at pulling a rope. Their task is to pull the rope strong enough in order to win by pulling more than their opponent. Participants are told they are facing human players that are found at the other side of the rope in the next room, however they will be playing against programmed agents. Participants will be using VR gloves and will be holding a real life rope that corresponds to the virtual rope. Each trial will last 2 minutes and there will be a noise at the start of each trial.


Experiment Design
The experiment has a within subjects design varying the appearance of the opponent avatar in terms of strength (normal and strong). A possible confound is adding aggression cues to strong avatars in order to increase their perception of strength. It is unclear whether aggression or strength could elicit different responses. Adding aggression as another independent variable might help decouple any possible effect.

Another issue is that participants will need to form a baseline opinion of what is a strong-looking avatar. As such, ordering might have too big an effect on the results of the experiment. Their assumption about the designs of the avatars could change. To mitigate this there are several possibilities:
Experimenter makes comment about how avatar design turned out as inspired by previous work for the Proteus effect: a general comment about all designs “We tried to design different avatars for everyone but some ended up a bit too buff and others not so much”, a comment before each avatar: “This one ended up a bit too strong”. A downside of this approach might be an additional effect from priming the participants.
Users can have an overview of all avatars in the beginning before they start where they see all the players and then wave at each other, after which they disappear from view and a trial with the first player begins. 

A weak appearance condition can be added to further investigate performance change. Putting a weak and strong avatar next to each other will make this trait more salient and help participants establish a baseline for this condition. The avatar designs will be chosen after a short survey online investigating which avatars users think look the strongest, most aggressive. 
