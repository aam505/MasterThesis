
\section{Methodology}
\label{section:methodology}
To investigate whether an agents' appearance can determine people to pull stronger, we ran a study where participants played tug-of-war with an opponent in VR.\\
The experiment has two parts: a survey and a user study. We ran a survey in order to determine perceptions of strength and intimidation in avatar design. We chose the avatars for the user study based on the results from the survey, where we measured perceived strength and intimidation. Please see section \ref{section:survey} for more details about the survey and choice of avatars. The user study represents the actual empirical study of our experiments. We presented this research as a VR gaming study. This setup allowed us to vary the opponents' appearances in a natural way. Realism is important to sustain the illusion that some force could be activating on the rope. Furthermore, it allows us to take objective measures of performance without interfering with the users' experience.

\subsection{Research Aims}
We aimed to give participants the illusion that the rope is being pulled back harder by stronger opponents. Equivalently, we expected people to feel the rope being pulled less by weaker opponents. If participants assumed these expectations, we hypothesized they would also perceive pulling harder for strong opponents, and less for weaker ones. Additionally, we measured the actual force of the pull to verify performance differences.
 \\
Our independent variable is the appearance of the opponent. Each opponent was randomly assigned a condition of this variable, from weak-looking to strong-looking. We investigated if participants perceived any change in rope-pulling from one trial to another. We state our first hypothesis:
\begin{quote}
\textbf{H1}: \textit{Participants will perceive changes in rope-pulling force between opponents.}
\end{quote}
To measure perceived changes in rope pull, we introduce a dependent variable, \textbf{challenge}. We asked participants to rate how challenging each rope-pull was after the respective trial. Additionally, we left it to the participant to interpret what \textit{challenging} means.\\ 
We introduce \textbf{perceived pull} as a subjective measure of how much participants thought they pulled. We measured both variables on a 5-point Likeart scale. For each condition, we also took objective measurements of force. We used a force meter to detect the maximum force each participant pulled per trial. This constitutes the the third dependent variable, \textbf{force}. We expect users to react to the appearance of their opponent and pull harder for stronger-looking opponents. We state the hypotheses of these variables:\\

\textbf{H2}: \textit{Participants will use more force for stronger-looking opponents.}\\

\textbf{H3}: \textit{Participants will report pulling harder for stronger-looking opponents.}\\

\textbf{H4:}\textit{ Participants will find stronger-looking opponents more challenging.}\\
\\
For H2, we use the maximum pull per trial as a measure of \textit{more force}. 

\subsection{Experiment Design}
\label{subsection:ExperimentDesign}
We had a gender-matched, within-group experimental design for the user study. We independently varied the agent's appearance on a 5-point scale from weak-looking to strong-looking. Participants played five trials of rope-pulling with the five avatars chosen from the survey. The avatars were meant to display various degrees of strength and intimidation, compounded in a final weighted score. We chose the weakest (condition 1) and strongest (condition 5) male and female avatars. For the remaining three, we chose avatars in a low-average, average and high-average strength condition. Further details about the avatars chosen for the experiment are presented in section \ref{section:surveyResults}. For each trial, we measure the maximum pull force in kilograms for each participant, which constitutes our dependent numeric variable. We further measure perceived pull and perceived challenge on a 5-point Likert scale, with constitute our two ordinal dependent variables.\\
Participants completed a post-experimental survey, gave feedback for each rope-pulling trial in-game and had a short chat with the experimenter at the end.\\
Between each rope-pull, participants rated four statements about the previous round on a 5-point Likert scale. They were presented with a panel of these questions in VR, and were asked to read the questions out-loud and give their answer to the experimenter. These questions are the following:
\begin{itemize}
\label{enum:panelQuestions}
\itemsep0em
\item \textbf{Q1}: I felt the virtual rope was realistic. Rating: 1 (\textit{Fully disagree}), to 5 (\textit{Fully agree});
\item \textbf{Q2}: It looked and felt like I was the one holding the rope. Rating: 1 (\textit{Fully disagree}), to 5 (\textit{Fully agree});
 \item \textbf{Q3}: How much did you pull the rope? Rating: 1 (\textit{Not at all}), to 5 (\textit{Very much});
\item \textbf{Q3}: How challenging was this round? Rating: 1 (\textit{Not at all}), to 5 (\textit{Very challenging}).
\end{itemize}
Q1 refers to rope realism, Q2 captures rope agency, Q3 refers to perceived pull and Q4 refers to challenge. We will use these terms to refer to the categories of these questions in the results section. Q3 and Q4 are two dependent variables. We present results of Q1 and Q2, however their main purpose was to give participants the impression we are evaluating rope performance.\\
\\
For the post-experimental survey, we measured participants' subjective experience on three levels: body ownership, presence and co-presence measures. 
To measure presence, we retained 5 items from the igroup presence questionnaire (IPQ): \footnote{http://www.igroup.org/pq/ipq/index.php}
\begin{itemize}
\label{enum:presenceQuestions}
\itemsep0em
    \item \textbf{Q1}: How aware were you of the real world surrounding while navigating in the virtual world? (i.e. sounds, room temperature, other people, etc.)? Rating: 1 (\textit{Not aware at all}), to 5 (\textit{Extremely aware});
    \item \textbf{Q2}: How real did the virtual world seem to you? Rating: 1 (\textit{Not real at all}) to 5 (\textit{Completely real});
    \item \textbf{Q3}: How much did your experience in the virtual environment seem consistent with your real-world experience ?  Rating: 1 (\textit{Not consistent}) to 5 (\textit{Very consistent});
    \item \textbf{Q4}: I felt present in the virtual space. Rating: 1 (\textit{Fully disagree}) to 5 (\textit{Fully agree});
    \item \textbf{Q5}: In the computer generated world I had a sense of ``being there''.  Rating: 1 (\textit{Not at all}) to 5 (\textit{Very much}). 
\end{itemize}
The IPQ questionnaire measures presence on three levels: spacial presence, involvement and experienced realism. From our retained questions, Q1 refers to involvement, Q2 and Q3 capture realism, Q4 spacial presence and Q5 refers to general presence.
\\
The survey items for body ownership were taken from \cite{argelaguet2016role}. For our study, we replaced the term \textit{hand} with \textit{arm}, and changed the 7-point rating scale to a 5-point one. The following questions were retained:
\begin{itemize}
\label{enum:ownershipQuestions}
\itemsep0em
    \item \textbf{Q1}: I felt as if the virtual arm moved just like I wanted it to, as if it was obeying my will. Rating: 1 (\textit{Fully disagree}), to 5 (\textit{Fully agree});
    \item \textbf{Q2}: I expected the virtual arm to react in the same way as my own arm. Rating: 1 (\textit{Fully disagree}), to 5 (\textit{Fully agree});
    \item \textbf{Q3}: I felt that the interaction with the environment was realistic. Rating: 1 (\textit{Fully disagree}), to 5 (\textit{Fully agree});
    \item \textbf{Q4}: I felt like I controlled the virtual arm. Rating: 1 (\textit{Fully disagree}), to 5 (\textit{Fully agree});
    \item \textbf{Q5}: I felt as if the virtual arm was part of my body. Rating: 1 (\textit{Fully disagree}), to 5 (\textit{Fully agree});
    \item \textbf{Q6}: I felt as if the virtual arm was someone else’s. Rating: 1 (\textit{Fully disagree}), to 5 (\textit{Fully agree}).
\end{itemize}
With respect to co-presence/social presence, we retained three questions from \cite{nowak2003effect}, changed their scale to a 5-point one and replaced the term \textit{interaction partner} with \textit{opponents}. They questions are:
\begin{itemize}
\label{enum:copresenceQuestions}
\itemsep0em
    \item \textbf{Q1}: My opponents were intensely involved in our interaction. Rating: 1 (\textit{Fully disagree}), to 5 (\textit{Fully agree});
    \item \textbf{Q2}: To what extent did you feel able to assess your opponents’ reactions?. Rating: 1 (\textit{I was unable}), to 5 (\textit{Their reactions were clear});
    \item \textbf{Q3}: To what extent was this like you were in the same room with your opponents? Rating: 1 (\textit{Did not feel in the same room}), to 5 (\textit{Felt completely in the same room}).
\end{itemize}
Additionally, the participants had to rate the appearance of the avatars on the same scale as in the initial appearance survey. We did this in order to verify our assumptions with respect to the hypothesis, that the opponents users faced were indeed perceived as strong/intimidating. These questions were in the final part of the post-experimental survey:
\begin{enumerate}
\itemsep0em 
\item This avatar looks attractive.
\item This avatar looks strong.
\item This avatar looks intelligent.
\item This avatar looks intimidating.
\end{enumerate}
The avatars were randomized for each participant at the start of the experiment. This study was ran alongside another study by researchers within the department. We made a common call for participants to take part in a \textit{VR Games Study}. Their task was to assess two virtual reality games and give researchers feedback about their experience. The two games were Tug-of-war VR and Whack-a-mole VR. At the end of the experiment, participants had a short recorded chat with the experimenter. Participants were asked to give feedback for the games and, in addition, we checked their awareness about the true purpose of the experiment.
\\
We frame our research in the context of a competitive strength task between a user and a perceived agent in virtual reality. A competitive set up allows participants to face their opponents directly in a physical task. Furthermore, it constitutes a realistic interaction for a game setup. Players were told they were testing a rope pulling VR game and they would be asked for feedback and suggestions to improve the game.
Their reported task was to face several different \textit{opponents} in VR, compete by pulling a rope and win. The player was able to see their hands in the virtual environment holding the rope. They were instructed to keep their hands on the rope at all times. Their fingers were not animated and the grip was fixed on the rope, as such letting go of the rope would result in breaks of presence and ownership. We do not make the aim of our research transparent to avoid any possible biases, such as the observer expectancy effect, or induce powerful demand characteristics.
\\
Participants see a countdown accompanied by sounds for each visual element being displayed. They are told to start pulling when they see \textit{Start}, and stop pulling when they see \textit{Stop}. When participants start pulling, they also see their opponent pulling and feel the resistance increase on the rope as they go on. Through this flow of events we synchronize, visual, audio and haptic feedback in order to give users the impression of agency and realism. 