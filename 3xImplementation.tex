
\section{Implementation}

For the experiment we used a Predator Helios 300 laptop with an Nvidia GeForce GTX 1060, VRREADY graphics card. VE immersion was achieved by using a HTC Vive headset. \todo{add specs here}
Players' hand movements were tracked by the Noitom HI5 Vive tracker gloves \footnote{https://hi5vrglove.com/}. The glove uses a wrist-mounted Vive tracker to generate users' presumed hand position. 

The tug-of-war game was implemented in Unity3D and integrated with VR through SteamVR. Auditory feedback was provided for the countdown before the rope-pulling in order to increase the appearance of a gaming situation. Sounds were provided from the laptop speakers which was located behind the black curtains.

The avatars of the players were designed from the Morph Character System (MCS) female and male humanoid avatars  \footnote{https://connect.unity.com/p/morph-3d-morph-character-system-mcs}. The models were edited in Autodesk Maya \footnote{https://www.autodesk.com/products/maya/overview} to remove parts of the mesh so that only the upper torso and arms were displayed. The original skin material was replaced with a custom design with less detailing to remove any uncanny valley effects \cite{geller2008overcoming}. 

In the game players could see their left and right hand holding the rope which comprised their embodied avatar. \todo{add male and female hand picture}. The virtual grip of the hands on the rope was taken from a real hand grip of the rope using the HI5 gloves. The finger movement was disabled as the gloves were easily magnetized by the set up and would malfunction shortly after the beginning of the experiment. To prevent loss of ownership from random finger movements, players were told to maintain the same grip on the rope at all times during the experiment. Before starting, users would undergo the calibration of the gloves provided with the HI5 glove set up to maintain accurate hand orientation and position. Despite calibration efforts and manual adjustment trials, the  undesirable finger movement and often inaccurate hand orientation and position proved to be the most difficult set back of this project. This was most challenging when integrating the gloves within an inverse kinematics VR algorithm to display user arm movement in a natural manner.

We used Final IK VR \footnote{https://assetstore.unity.com/packages/tools/animation/final-ik-14290} to display naturalistic avatar body movements. The positions of reference for the algorithm were the left and right hand trackers, together with the headset position. 
This implementations dies not always congruently display the position of the virtual arm and user arm. It is, however, an acceptable trade-off due to the lack of manual elbow tracking and full body tracking. Furthermore, due to the haptic sensation from actually holding a rope and the competitive aspect of the game, users were mainly focusing their attention away from their hands and on their opponent. \todo{explain down the line where they looked and say see chapter idk}.
The virtual rope was implemented using the Obi Rope asset \footnote{https://assetstore.unity.com/packages/tools/physics/obi-rope-55579}. Its settings were adjusted manually to simulate real rope movements through a trial-and-error approach. Please refer to appendix A for the chosen rope settings. \todo{add rope settings to appendix}. The rope texture was modeled to the real rope users were holding \todo{picture of rope maybe}. 

\subsection{Agent Design}

The virtual agents were created in Unity using the Unity Multipurpose Avatar 2 (UMA) asset \footnote{https://assetstore.unity.com/packages/3d/characters/uma-2-unity-multipurpose-avatar-35611} and o3n UMA Races \footnote{https://assetstore.unity.com/packages/3d/characters/humanoids/o3n-male-and-female-uma-races-102187}. We chose this library because the UMA character system has a high degree of avatar customization and can be generated with relative ease. Programmers are able to vary body part sizes and, among many others, ear and brow rotations. 

We varied perceived strength in the design of the agents and used UMA DNA settings to to generate a distribution of agents varying from strong looking to average and weak looking. UMA Dna represents a dictionary used to customize Unity multipurpose avatars (UMAs). Appearance can be changed on a scale of 0 to 1, with higher values being used to make adjustments more salient. A downside of this character system is that the magnitude of these features does not vary linearly and can differ between races and genders. For example, female muscle size cannot be increased on the same scale as male muscle size. Due to this, male agents have higher perceived strength. This can also be observed in the results of our survey in the next chapter. Usually, UMAs in the strong conditions had strength-signaling dna parts closer to 1. Their values decreased to 0.5 for agents in the normal conditions and below 0.5 for agents in the weak condition. We considered 0.5 as the default average dna setting. -- they were not linear so they had to be adjusted after all agents were generated with these values some appeared too uncanny or unnatural...


As strength cues we used muscle tone and fitness, together with adjusted facial proportion as seen in \cite{windhager2011geometric}. Skin, eye and hair color, together with outfit designs were kept as constant as possible between races and genders. For the upper body clothing, the agents had a sleeveless shirt to increase the visibility of the arm muscles. 
With respect to facial manipulations, we varied facial width and jaw size which are known to be associated with testosterone in men \cite{lefevre2013telling}. We decreased the size of the lips and eyes in the strong conditions to increase face-width ratio. While height and age are the best predictors for female perceived strength \cite{sell2008human}, we applied the same variations to the female avatars. We do not adjust height for the agents displayed in the survey images, however we manipulate height in the actual experiment, increasing height for agents in the strong condition.  

As strength is closely related with dominance and aggression, we added variables meant to increase intimidation such as tattoos, military hairstyles and manipulated outfit colors. As in previous Proteus effect literature and associated body of work related to social psychology \cite{yee2009proteus,pena2009priming}, we used black clothing to elicit more aggressive attitude perceptions. Agents in strong conditions had black upper-body clothing, those in medium conditions had gray and for weak agents we used white. While male tattoos are dominantly viewed, female tattoos generate mixed responses from observers \cite{wohlrab2009perception}. Agents were gender matched to avoid any tensions or similarity biases that would arise from cross-gendered evaluations. Please see Appendix B for the agent designs presented in the survey
We designed the avatars within the scope of the allowed intervals for each variable set by UMA. While we tried to maintain as little difference between males and females, not all variables offered the same magnitude of variation and there were various differences. Some examples of these are: females could not have their muscle mass increased as much as men, increasing lower muscle mass for women inflated the leg muscles in an unnatural way, difference in colors due to mesh materials. Further, we use the UMA Expression Player to adjust facial expression to an adequate neutral state as the eyebrow designs were unsatisfying in their default state. This is because the eyebrows were too high and rotated outwards giving the impression of a surprised, submissive look. We pushed the brows in to form a neutral expression.
Using stratified sampling, we provide a good range of looks for the male avatars and an adequate one for female avatars due to constraints imposed by the library. Through the survey, we aim to make an informed choice about the avatars for the experiment and provide designs that elicit realistic perceptions of intimidation and strength. Please see Appendix A for a full list of avatar design choices based on the UMA library.
