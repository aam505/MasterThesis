
\section{User Study}

\subsection{Experiment Design}

\subsection{Setup}
-
- record with obs sreen and game view of participants

- record video and audio

Setting 
To sustain the plausibility illusion, the setting must be consistent with a tug of rope situation.
Feeling rope increases plausibility
No dis tractors in the scene except for something between the avatars
The illusion of a hole may increase anxiety, fear or trigger a more realistic response through increased arousal.



The rope will be tied to a force gauge that will measure the pull force of each participant. The gouge is filmed and recorded. After the experiment the force is manually recorded by the experimenter and plotted at set intervals. To provide a minimum force, the rope will be attached to a strong spring, however the actual pull of another player will not be simulated. 
Afterwards we conduct a survey in which participants rate the self-perceived force they had for each condition, and how much they felt their opponents pulled. We also measure how strong they think each opponent looked. 
At the end we ask about the participant’s experience with the game in a short 2 minute interview and if they have any suggestions to improve it. Finally we ask the participants what they think the purpose of this experiment was in order to discard participants who showed a high awareness of the experimental aim. In the end we debrief users and tell them about the intended purpose of the experiment.

\subsection{Procedure}
\subsection{Piloting}
While participants must believe they are evaluating a game, it is also important that participants believe that they are playing against other humans. We provide two possible design frameworks to increase this believably. 

We ran several pilots to see if it is feasible that participants believe they are co-located with other players and that at the end of the rope there is another human. We investigate whether participants actually feel the rope is being pulled more in one condition despite this not being implemented in any way. The question is whether appearance change in virtual reality can provide a working illusion and create a belief that is manifested in a measurable way, namely in the performance of players.
If we cannot convince players they are co-located, then we frame our experiment in the context of designing a system to simulate force in a distributed way. Participants will be told we made a device and we are exploring its feasibility to be used in games as a distributed force transmitter. 
To increase the belief in agency, participants are told to wave at each player before starting the game. Avatars making some form of acknowledgement of the other player will increase the believably they have agency. 


\subsection{Setup Adjustments}

\subsection{Results}
\subsubsection{Quantitative Results}
\subsubsection{Qualitative Observations}


\subsection{Discussion}
