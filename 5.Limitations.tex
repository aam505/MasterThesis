\section{Limitations}
\label{section:limitations}
While our study showed great potential in supporting H1, we had many technical and functional limitations.\\
First of all the spring we used to provide resistance absorbed some amount of force and the data we gathered is not the actual pull. Changes in the physical structure of the spring could  also have altered our measurements between participants. For example. one of the participants stretched the spring so much that it had to be replaced. However, we believe that, if the spring had not been integrated, some participants would have pulled over the maximum measurable force of the meter. Ultimately, this is a constraint of the measuring instrument. \\
Furthermore, the force data we gathered does not reflect the whole story of how participants pulled. We measured the maximum force, however other measures such as the longest value displayed on the monitor or the average value during a trial might reflect participants' performance better. Other suggested measuring techniques would be to take intervals at the end of the countdown and at the start of the countdown to see how fast participants got to the maximum value of that trial. Unfortunately, as participants did not continuously pull the rope, our finding suggest fixed interval measures would be unreliable. 
\\
Another limitation is that he rope was not tied to a fixed object and, as such, participants were asked to keep their legs still when pulling. Restraining participants in this manner may have induced higher cognitive load and prevented them from acting realistically. While we had extensive piloting to verify the setup, a few participants still showed great force and pulled the force meter too much. These situations affected their subsequent pulls as they noticed they had to restrain themselves.
