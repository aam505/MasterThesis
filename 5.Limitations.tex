\section{Limitations}
\label{section:limitations}
- spring absorbs force a participant stretched the spring too much and it was replaced\\
- force meter accuracy for the digital display
- person error prone?
-resolution and realism all had same movements but varying so many things would allow less control over the independent variables research wise, it would be difficult to establish causation,//
-ppl unsure about pulling the first round and there is some effect//
- force got maxim for whatever minimum amount of time other strategies can be used to measure with a force meter that is not digital - like the average value across all that time, the longest value for how many seconds. etc.
- there could be some effect that people held medium values for longer idk
-challenge - what is challenging what they rated with if they rated it by looks then it is bad. MMaking it implicit and explicit has advantages and disadvantages. For example cognitive load focusing only on the pull ...\\
-change ratings allow and mention some did some not

0-The downside of this is that some participants mentioned they experience cognitive load as they were trying not to move their legs during the pulls. They were focusing on that instead of focusing on pulling as hard as they can. To solve this, we can use a set up in which the ropes are tied to walling.

-participants misbehave pilot pilot to design the experiment against misuse

- low resolution of vr headset i recommend something to make people stay closer to the avatars  and allow them to see them better