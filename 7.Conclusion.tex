\section{Conclusion and Future Work}

Virtual reality's immersive capabilities have been successfully applied in fields such as entertainment, military training and phobia management in order to elicit real-world human experiences. Bowman and McMahan examine some of these successful VR ventures and suggest their aspired fidelity to real life is what sets them apart \cite{bowman2007virtual}. In the long term, the goal of many VR researchers and enthusiasts seems to be engaging as many of the human senses as possible to replicate real-life sensory experience with high accuracy. Our findings inform the design of 
\\
For this instance of the experiment, we do not manipulate avatar appearance. In VR, people will be able to see their arms holding the rope. However, we acknowledge the significant effect of embodiment [citations] and postulate that a physical transformation of the self, coupled with an appropriate context, would generate a better illusion.
\\
Discuss and reason about social effects in VR with respect to possible applicabilities in
the industry and for future research.
- force got maxim for whatever minimum amount of time other strategies can be used to measure with a force meter that is not digital - like the average value across all that time, the longest value for how many seconds. etc.
\\
Our

We hypothesized that seeing stronger opponents would determine people to pull harder. As P13 mentioned, it seemed like a \textit{a natural response}. Perhaps more surprising than supporting this hypothesis, is that our experiment showed inconclusive results and great individual variation. Our main findings suggest that most participants perceived variations in the force acting on the rope. This is reflected by the fact that they reported different challenges across trials, giving support to H1. Furthermore, we also observed breaks in the illusion when participants' expectations were not met and, instead, the system provided incongruent motor and visual feedback.
\\
It appeared that users were driven by curiosity and acted in unexpected ways when interacting with the system, probing and testing its capabilities. We strongly advise designers to provide as many contingencies for they system as they can. Otherwise, breaks in realism and ownership can decay the illusions generated in VR. We posit sustaining these illusions is also required in order to maintain high engagement and keep users motivated to use the system. High behaviour and physical realism might not be necessary, however it seemed to provide limits in which these kinds of illusions can occur \cite{slater2009place}. \\
Furthermore, we observed that physical objects which provide continuous haptic feedback might serve as distractors and mitigate visual inaccuracies. However, we believe that affording realistic interaction with the these objects is key to sustaining presence and engagement. More research is required to support our observations.
\\
The virtual human opponents seemed to be the focal point of the interaction. Participants had expectations that were not met about the interaction with the avatar which seemed to place doubts about the realism of the whole system in their minds. We believe it is useful to provide as few boundaries for the interaction as possible. Due to the novelty of the situation, people might behave differently. \\
Researchers have shown body ownership occurs with dramatically different appearances, with larger or even multiple body parts, dubbed as \textit{Homuncular flexibility} \cite{normand2011multisensory, won2015homuncular}. Evidently, virtual reality has demonstrated its ability to allow human perception to transcend its normal bounds. However, we believe more research is required to explore how far VR can go in altering lived experiences. While most research has focused on internal perceptions, we have shown evidence that virtuality is capable of altering people's beliefs about the real world. As VR technology becomes more advanced, it is important for designers to know the extent of the kind of experience they want to create with their product. As the line between reality and virtuality becomes blurred, the ethical implications are clear. If physical illusions can occur with such a simple setup, generating complex illusions is limited only by the creativity of the designer.
 
 