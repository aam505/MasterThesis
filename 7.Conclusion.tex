\section{Conclusion}
\textcolor{red}{DRAFT}\\
Referring to stronger avatars 
Okay… I should pull stronger, it’s a natural response. P13 - seems like a natural response, however the context was not something familiar and it determined people to have exploratory actions like testing the environment or hesitating.etc

Illusions are feasible, seem to be mitigated by realism and decay in time because of sensory adaptation. meet expectations to sustain illusion like Slater said \cite{slater2009place}. Users interacting with the system might have a playful attitude, probe and test the environment. Provide contingencies for unexpected participant behavior. Otherwise, breaks in realism and ownership can decay the illusion. High behavior and physical realism might not be accessory but it provides limits in which in can occur \cite{slater2009place}. Most important to afford realistic interaction with the haptic object people interact with. The social aspect of the technology seemed to be the focal point of the interaction. Participant had expectations that were not met about the interaction with the avatar which seemed to place doubts about the realism of the system in their minds. Either way, useful to provide as few boundaries for the interaction as possible. We posit it might be useful to include some haptic feedback in order to generate more body ownership when degrees of freedom of the actual body or realism cannot be matched. We also expect people behaved differently because of the novel situation that they were in. They showed reluctance to pull the rope. A demonstration by the experimenter with physical actions can be useful to clarify the instructions.