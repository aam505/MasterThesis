\section{Conclusion and Future Work}

We hypothesized that seeing stronger opponents would determine people to pull harder. As P13 mentioned, it seemed like a \textit{a natural response}. Perhaps more surprising than supporting this hypothesis, is that our experiment showed inconclusive results and great individual variation. Our main findings suggest that most participants perceived differences in the force acting on the rope. This is reflected by the fact that they reported different challenges across trials, giving support to H1. Furthermore, we also observed breaks in the illusion, when participants' expectations were not met and, instead, the system provided incongruent motor and visual feedback.
\\
The virtual human opponents seemed to be the focal point of the interaction. Participants had expectations that were not met about the interaction with the avatar, which seemed to place doubts about the realism of the whole system in their minds. For example, they expected the opponent to react to their feedback. Opponents were visually shown to push and pull with the beginning and end of the countdown. We did not provide contingencies, however, if participants strayed from the standard flow of the interaction. When they did, they found their expectations were not met. In order to sustain realism, we believe it is useful to provide as few boundaries for the interaction as possible. This means that designers should provide as many contingencies as they can in their virtual experiences. Assuming that users will follow a standard path does not seem feasible.
 \\
When users' expectations are not met, breaks in realism and ownership appear to occur. This eventually appears to decay the illusions generated in VR. These illusions that the virtual world behaves like the real world are meant to elicit realistic behaviour from users. Additionally, we believe sustaining these illusions is also required in order to maintain high engagement and keep users motivated to use the system. High behaviour and physical realism might not be necessary, however it seems to provide limits in which these kinds of illusions can occur \cite{slater2009place}. While we provide some findings with respect to decays of illusion, more research is necessary to examine the underlying cognitive mechanisms and their effect on the overall user experience. We suggest that this process can be examined with controlled experiments where breaks are induced by the experimenter. 
 \\
Embodiment is one of the many illusions generated by the virtual environment. In the present work, we do not manipulate the appearance of the users' avatar. However, we acknowledge the significant effect of embodiment on users' perceptions and postulate that a physical transformation of the self, coupled with an appropriate context, would generate a better illusion. Providing a contrast between the opponent and the agent could highlight various stimuli, such as making the opponent's strength more obvious. However, the interplay between the observer's appearance and the virtual representation of others is not fully understood. Decoupling the two remains a challenge for future researchers.
\\
Researchers have shown that body ownership occurs with dramatically different appearances, with larger or even multiple body parts, dubbed as \textit{Homuncular flexibility} \cite{normand2011multisensory, won2015homuncular}. Evidently, virtual reality has demonstrated its ability to allow human perception to transcend its normal bounds. However, we believe more research is required to explore how far VR can go in altering lived experiences. While most research has focused on internal perceptions, we have shown evidence that virtuality is capable of altering people's beliefs about the real world. As VR technology becomes more advanced, designers need to know the extent of the kind of experience they want to create.
\\
Additionally, we observed that physical objects which provide continuous haptic feedback might serve as distractors and mitigate visual inaccuracies. However, we believe that affording realistic interaction with these objects is key to sustaining presence and engagement. Our findings suggest that haptic feedback could be used to induce stronger ownership illusions. This is especially desirable when visual cues are lacking. More research is required to support our observations.
\\
 Virtual reality's immersive capabilities have been successfully applied in fields such as entertainment, military training and phobia management in order to elicit real-world human experiences. Bowman and McMahan examine some of these successful VR ventures and suggest their aspired fidelity to real life is what sets them apart \cite{bowman2007virtual}. In the long term, the goal of many VR researchers and enthusiasts seems to be engaging as many of the human senses as possible to replicate real-life sensory experience with high accuracy. Ultimately, our findings inform the design of realistic VR applications. As mixed reality becomes more and more popular, the seamless integration of haptic feedback appears to be a necessity. Therefore, generating physical perceptual differences which comply with visual feedback.
 