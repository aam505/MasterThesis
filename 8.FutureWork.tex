\section{Future Work}
\textcolor{red}{DRAFT}\\

Virtual reality's (VR) immersive capabilities have been successfully applied in fields such as entertainment, military training and phobia management in order to elicit real-world human experiences. Bowman and McMahan examine some of these successful VR ventures and suggest their aspired fidelity to real life is what sets them apart \cite{bowman2007virtual}. In the long term, the goal of many VR researchers and enthusiasts seems to be engaging as many of the human senses as possible to replicate real-life sensory experience with high accuracy. Our findings inform the design of 

For this instance of the experiment, we do not manipulate avatar appearance. In VR, people will be able to see their arms holding the rope. However, we acknowledge the significant effect of embodiment [citations] and postulate that a physical transformation of the self, coupled with an appropriate context, would generate a better illusion.

Discuss and reason about social effects in VR with respect to possible applicabilities in
the industry and for future research.
- force got maxim for whatever minimum amount of time other strategies can be used to measure with a force meter that is not digital - like the average value across all that time, the longest value for how many seconds. etc.

Our results show most participants detected variations in pulling. However the way in which they pulled was not consistent with our hypotheses, despite qualitative feedback. -detail In what follows we give an overview of our findings, explain the limitations of our study and give advice about using physical objects in VR to generate illusions. It appears physical illusions seem feasible even in a low-fidelity environments. However, we posit that some degree of behavioural and physical realism is desired, especially to sustain these perceptual overrides. If physical illusions can occur with such a simple setup, generating complex illusions is limited only by the creativity of the designer.
\\